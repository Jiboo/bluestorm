\documentclass[10pt,a4paper]{article}
\usepackage[utf8x]{inputenc}
\usepackage{ucs}
\usepackage{hyperref}
\usepackage{listings}
\usepackage{amsmath}
\usepackage{amsfonts}
\usepackage{graphicx}
\usepackage[frenchb]{babel}
\usepackage{amssymb}
\hypersetup{
pdfpagemode=UseOutlines,      % UseOutlines, UseThumbs, None, FullScreen : agencement au démarrage
pdfstartview=Fit,             % Fit, FitH, FitB, FitBH : vue de la page au départ (pleine largeur...)
pdffitwindow=true,            % bool: Maximiser
%pdfpagelayout=TwoColumnsRight,% SinglePage, TwoColumnsRight/Left, OneColumn : affichage des pages
pdftoolbar=true,              % bool: Affichage de la barre d'outils
pdfmenubar=true,              % bool: Affichage de la barre de menus
bookmarksopen=false,          % bool: Dépliement des signets
bookmarksnumbered=true,       % bool: Numérotation des signets
colorlinks=true,              % bool: Liens colorés
pdfauthor={Beno\^it Caruso, Jean-Baptiste Lepesme},     % Auteur
pdftitle={Cahier des charges.},    % Titre
pdfcreator=PDFLaTeX,          % 
pdfproducer=PDFLaTeX,         %
linkcolor=blue,               % Couleur des liens
urlcolor=blue,                %             url
anchorcolor=black,            %         du texte
citecolor=green,              % Couleur de citation 
frenchlinks=true,             % bool: Utiliser des petites majuscules pour les liens, plutôt que de la couleur
pdfborder={0 0 0}             % Ne pas encadrer les liens
}
\title{Cahier des charges}
\author{Jean-Baptiste Lepesme, Benoit Caruso}
\begin{document}	
	\maketitle%titre
	\tableofcontents%table des matières
	\newpage%saut de page
	
	\section{Actions a r\'ealiser}
		Dans un premier temps, nous devons prendre en main les outils de programmation pour le Lego Mindstorms et faire une application Android permettant de piloter le robot.\\
		Dans un second temps, il faudra implémenter une pile UPnP sur le robot, et lui implémenter des services permettant de le piloter via ce protocole.
		\subsection{Decouvrir le syst\`eme}
			\begin{tabbing}
			Il f\=audra dans un premier choisir la forme du robot, qui permetra le plus de possibilitées :\\
			\>	$ \bullet $ Humanoide\\
			\>	$ \bullet $ Vehicule a pince\\
			\>	$ \bullet $ Non mobile (bras)\\
			\end{tabbing}
			Ensuite, il faudra installer et configurer les outils pour développer et programmer le robot.
			%todo liste des outils utilisés ?
			
		\subsection{Experimenter avec l'API LeJOS}
			LeJOS est un firmware customiz\`e qui a \'et\`e install\'e sur le robot afin d'avoir une machine virtuelle Java et des API bas et haut niveau pour contr\^oller le robot.
		\subsection{Faire une application Android : Bluestorm}
			Une fois le fonctionnement du robot validé, nous développerons un serveur bluetooth compris dans le firmware afin de piloter directement le robot en bluetooth.\\%nous piloterons le robot via un serveur Bluetooth compris dans le firmware.\\
			Voir "Appendix 1-LEGO MINDSTORMS NXT Communication protocol" et "Appendix 2-LEGO MINDSTORMS NXT Direct commands" du Bluetooth Developer Kit disponible sur le site officiel de Lego Mindstorms.\\
			\subsubsection{Description de l'application}
				\begin{tabbing}
				Pilo\=tage\= du \=robot (surement un véhicule avec pinces) :\\
				\>	$ \bullet $ Utilisation du gyroscope :\\
				\>	$ \bullet $ La vitesse de rotation de chaque train dépend de l'orientation du téléphone.\\
				\>	$ \bullet $ Si le téléphone ne possède pas de gyroscope :\\
						\>\>L'application proposera 4 boutons :\\
						\>\>\>	$ \bullet $ Marche avant et arrière\\
						\>\>\>	$ \bullet $ Rotation (sur place) dans le sens horaire et anti-horaire\\
				\>	$ \bullet $ Des boutons permettrons d'ouvrir et fermer la pince.\\
				\>	$ \bullet $ Il sera possible de visualiser :\\
				\>\>		La distance du prochain obstacle (via le capteur ultra-son)\\
				\>\>		Le niveau de batteries\\
				\end{tabbing}
		\subsection{Implémenter une pile UPnP dans LeJOS}
			Faire du robot un device UPnP permettant ainsi de pouvoir le contr\^oler en appelant des actions sur le robot.
			\subsubsection{Étapes a réaliser}
			\begin{tabbing}
			$ \bullet $	Ajouter la pile UPnP CyberLinkJava sur le robot\\
			$ \bullet $	Spéc\=ifier les descripteurs du robot\\
			\>$ \bullet $	Le robot sera composé de subdevices représentant les capteurs et actionneurs.\\
			$ \bullet $ Implémenter les devices et les exposer via CyberLink\\
			$ \bullet $ Développer un point de contrôle "web" en utilisant la fonctionnalités de présentation \\ de l'UPnP permettant de piloter le robot et d'afficher l'état des capteurs.\\
			\end{tabbing}
	\section{Le descripteur UPnP}
		La liste des actions possible pour l'UPnP:
		\begin{tabbing}
		$ \bullet $ Avancer/Reculer\\
		$ \bullet $ Rotation sur place\\
		$ \bullet $ Fermer/Ouvrir la pince\\
		$ \bullet $ Obternir le niveau de la batterie\\
		$ \bullet $ Emmetre un bip
		\end{tabbing}
		
		Les subdevices (capteurs et actionneurs) disposeront aussi d'actions selon leurs type:
		\begin{tabbing}
		$ \bullet $	Servos moteurs: Regler la vitesse du moteur\\
		$ \bullet $ Capteur de distance: Obtenir la distance de l'obstacle\\
		$ \bullet $ Capteur de couleur: Obtenir la couleur apercu\\
		$ \bullet $ Capteur de touche: Obtenir l'etat du capteur\\
		...
		\end{tabbing}

	\section{Planning}
		8 octobre : Cahier des charges finnalisé\\
		15 octobre : Premier mouvements du robot via Bluetooth\\
		29 octobre : Conception de l'application Android terminée\\
		26 novembre : Code de l'application Android terminée\\
		10 décembre : Le robot est un périphérique UPnP
\end{document}

