\documentclass{article}
\title{Cahier des charges}
\author{Jean-Baptiste Lepesme, Benoit Caruso}
\date{}
\begin{document}	
	\maketitle
	
	\tableofcontents
	
	\newpage
	
	\section{Actions a r\'ealiser}
		Prendre en main les outils de programmation pour le Lego Mindstorms et faire une application Android permettant de piloter le robot.
		Dans un deuxième temps, il faudra implémenter une pile UPnP sur le robot, et lui implémenter des services permettant de le piloter via ce protocole.
		
		% Remplacer les implemanter %
		
		\subsection{Decouvrir le syst\`eme}
			Il faudra dans un premier choisir la forme du robot, qui permetra le plus de possibilitées:
				- Humanoide
				- Vehicule a pince
				- Non mobile (bras)
			Dans un deuxieme temps il faudra installer et configurer les outils pour developper et programmer le robot.
		\subsection{Experimenter avec l'API LeJOS}
			LeJOS est un firmware customiz\`e qui a \'et\`e install\'e sur le robot afin d'avoir une machine virtuelle Java et des API bas et haut niveau pour controller le robot.
		\subsection{Faire une application Android}
			Une fois le fonctionnement du robot validé, nous piloterons le robot via un serveur Bluetooth compris dans le firmware.
			Voir "Appendix 1-LEGO MINDSTORMS NXT Communication protocol" et "Appendix 2-LEGO MINDSTORMS NXT Direct commands" du Bluetooth Developer Kit disponible sur le site officiel de Lego Mindstorms.
			\subsubsection{Les fonctions \`a r\'ealiser}
				Pilotage du robot (surrement un v\'eicule avec pince):
					- Utilisation du gyroscope:
						- La vitesse de rotation de chaque trains dépend de l'orientation du téléphone.
						- Si le téléphone ne possède pas de gyroscope:
							L'application proposera 4 boutons:
								- Marche avant et arri\`ere
								- Rotation (sur place) dans le sens horraire et anti-horraire
					- Des boutons permetrons d'ouvrir et fermer la pince.
					- Il sera possible de visualiser:
						La distance du prochain obstacle (via le capteur ultra-son)
						Le niveau de batteries
		\subsection{Impl\'ementer une pile UPnP dans LeJOS}
			Faire du robot un device UPnP
				Le robot contiendra des subdevices capteurs et actionneurs
					Ils contiendront des fonctions de bas niveau
				Le robot contiendra des fonctions haut niveau
	
	\newpage
				
	\section{Planning}
	
\end{document}

