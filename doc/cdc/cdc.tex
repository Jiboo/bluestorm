\documentclass{article}
\usepackage{listings}
\title{Cahier des charges}
\author{Jean-Baptiste Lepesme, Benoit Caruso}
\date{}
\begin{document}	
	\maketitle
	
	\tableofcontents
	
	\newpage
	
	\section{Actions a r\'ealiser}
		Prendre en main les outils de programmation pour le Lego Mindstorms et faire une application Android permettant de piloter le robot.
		Dans un deuxième temps, il faudra implémenter une pile UPnP sur le robot, et lui implémenter des services permettant de le piloter via ce protocole.
		\subsection{Decouvrir le syst\`eme}
			Il faudra dans un premier choisir la forme du robot, qui permetra le plus de possibilitées:
				- Humanoide
				- Vehicule a pince
				- Non mobile (bras)
			Dans un deuxieme temps il faudra installer et configurer les outils pour developper et programmer le robot.
		\subsection{Experimenter avec l'API LeJOS}
			LeJOS est un firmware customiz\`e qui a \'et\`e install\'e sur le robot afin d'avoir une machine virtuelle Java et des API bas et haut niveau pour controller le robot.
		\subsection{Faire une application Android : Bluestorm}
			Une fois le fonctionnement du robot validé, nous piloterons le robot via un serveur Bluetooth compris dans le firmware.
			Voir "Appendix 1-LEGO MINDSTORMS NXT Communication protocol" et "Appendix 2-LEGO MINDSTORMS NXT Direct commands" du Bluetooth Developer Kit disponible sur le site officiel de Lego Mindstorms.
			\subsubsection{Description de l'application}
				Pilotage du robot (surrement un v\'ehicule avec pince):
					- Utilisation du gyroscope:
						- La vitesse de rotation de chaque trains dépend de l'orientation du téléphone.
						- Si le téléphone ne possède pas de gyroscope:
							L'application proposera 4 boutons:
								- Marche avant et arri\`ere
								- Rotation (sur place) dans le sens horraire et anti-horraire
					- Des boutons permetrons d'ouvrir et fermer la pince.
					- Il sera possible de visualiser:
						La distance du prochain obstacle (via le capteur ultra-son)
						Le niveau de batteries
		\subsection{Impl\'ementer une pile UPnP dans LeJOS}
			Faire du robot un device UPnP permettant ainsi de pouvoir le controller en appelant des actions sur le robot.
			\subsubsection{Etapes a r\'ealiser}
				Ajouter la pile UPnP CyberLinkJava sur le robot
				Sp\'ecifier les descripteurs du robot
					Le robot sera composé de subdevice repr\'esantant les capteurs et actionneurs.
				Implementer les devices et les exposer via CyberLink
				Developper un point de controle "web" en utilisant la fonctionnalitée de présentation de l'UPnP permetant de piloter le robot et d'afficher l'état des capteurs
	
	\section{Le descripteur UPnP}
		La liste des actions possible pour l'UPnP.
		
		Avancer/Reculer
		Rotation sur place
		Fermer/Ouvrir la pince
		Obternir le niveau de la batterie
		Emmetre un bip
		
		Les subdevices (capteurs et actionneurs) disposeront aussi d'actions selon leurs type:
			Servos moteurs:
				Regler la vitesse du moteur (en \%)
			Capteur de distance
				Obtenir la distance d'un obstacle (en centimétres)
			Capteur de couleur
				Obtenir la valeur du capteur (Une enum represantant 6 couleurs et un état aucune couleur)
			Capteur de touche
				Obtenir l'état du capteur (booléen)
			...

	\section{Planning}
		8 octobre : Cahier des charges finnalisé
		15 octobre : Premier mouvements du robot via Bluetooth
		29 octobre : Conception de l'application Android terminée
		26 novembre : Code de l'application Android terminée
		10 décembre : Le robot est un périphérique UPnP
\end{document}

